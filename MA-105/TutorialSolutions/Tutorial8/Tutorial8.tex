\documentclass[14pt]{extarticle}
\usepackage{graphicx} % Required for inserting images
\usepackage{amsmath}
\usepackage{bbm}

\title{\vspace{-3cm}MA-105 Tutorial-8 Solutions}
\author{Daksh Maahor}
\date{September 2025}

\newcommand{\bb}{\mathbbm}

\begin{document}

\maketitle

\begin{enumerate}
    \item Compute the arc-length of the cycloid $\vec{r}(t) \ = \ (a(t-\sin{t}), \ a(1-\cos{t})); \ 0 \leq t \leq 2\pi$.
    \newline \hfill \newline
    Sol:
    \begin{align*}
        l \ &= \ \int_{0}^{2\pi} ||\vec{r}'(t)|| dt \\
        \text{Now, } \\
        \vec{r}(t) \ &= \ (a(t-\sin{t}), \ a(1-\cos{t})) \\
        \vec{r}'(t) \ &= \ (a(1-\cos{t}), \ a\sin{t}) \\
        ||\vec{r}'(t)|| &= \sqrt{(a(1-\cos{t}))^2 + (a\sin{t})^2} \\
        &= \ \sqrt{a^2(1+\cos^2{t}+\sin^2{t-2\cos{t}})} \\
        &= \ a\sqrt{2-2\cos{t}} \\
        &= \ a\sqrt{4\sin^2{\frac{t}{2}}} \\
        ||\vec{r}'(t)|| \ &= \ 2a\left | \sin{\frac{t}{2}} \right | \\
        \text{So, } \\
        l \ &= \ \int_{0}^{2\pi} 2a\left | \sin{\frac{t}{2}} \right | dt \\
        &= \ \int_{0}^{\pi} 4a |\sin{x}| dx \\ 
        &= \ 8a
    \end{align*}

    \item Parametrize the ellipse $\frac{x^2}{a^2}+\frac{y^2}{b^2} = 1$ and set-up the integral for the perimeter of the ellipse in terms of its eccentricity $e = \sqrt{1-\frac{b^2}{a^2}}$. Assume $b < a$.
    \newline \hfill \newline
    Sol:
    \begin{align*}
        \text{The parametrization} & \text{ of the ellipse } \frac{x^2}{a^2}+\frac{y^2}{b^2} = 1 \text{ is given by:} \\
        x \ &= \ a\cos{\theta} \\
        y \ &= \ b\sin{\theta} \\
        \theta \ &\in \ [0, 2\pi] \\
        \text{Hence, an ellipse can}& \text{ be described as the following function:} \\
        \vec{r}(\theta) \ &= \ (a\cos{\theta}, \ b\sin{\theta}) \\
        \text{Now, } \text{ } \text{ } \text{ } \text{ } \text{ } \text{ } \text{ } \text{ } \text{ }\text{ }\text{ }\text{ }\text{ }\text{ }\text{ }\text{ }\text{ }\text{ }\text{ }\text{ }\\
        \vec{r}'(\theta) \ &= \ (-a\sin{\theta}, b\cos{\theta}) \\
        ||\vec{r}'(\theta)|| &= \sqrt{(-a\sin{\theta})^2 + (b\cos{\theta})^2} \\
        &= \ \sqrt{a^2\sin^2{\theta}+b^2\cos^2{\theta}} \\
        \text{So,} \text{ }\text{ }\text{ }\text{ }\text{ }\text{ }\text{ }\text{ }\text{ }\text{ }\text{ }\text{ }\text{ }\text{ }\text{ }\text{ }\text{ }\text{ }\text{ }\text{ }\text{ }\text{ }\text{ }\text{ }\\
        l \ &= \ \int_{0}^{2\pi} ||\vec{r}'(\theta)|| d\theta \\
        &= \ \int_{0}^{2\pi} \sqrt{a^2\sin^2{\theta}+b^2\cos^2{\theta}} \ d\theta \\
        &= \ 4\int_{0}^{\frac{\pi}{2}} \sqrt{a^2\sin^2{\theta}+b^2\cos^2{\theta}} \ d\theta \\
        &= \ 4a\int_{0}^{\frac{\pi}{2}} \sqrt{\sin^2{\theta}+\frac{b^2}{a^2}\cos^2{\theta}} \ d\theta
    \end{align*}
    \begin{align*}
        \text{ }\text{ }\text{ }\text{ }\text{ }\text{ }\text{ }\text{ }\text{ }\text{ }\text{ }\text{ }\text{ }\text{ }\text{ }\text{ }\text{ }\text{ }\text{ }&= \ 4a\int_{0}^{\frac{\pi}{2}} \sqrt{\sin^2{\theta}+\left ( \frac{b^2}{a^2} - 1 \right )\cos^2{\theta} + \cos^2{\theta}} \ d\theta \\
        &= \ 4a\int_{0}^{\frac{\pi}{2}} \sqrt{1-e^2\cos^2{\theta}} \ d\theta \\
        &= \ 4a\int_{0}^{\frac{\pi}{2}} \sqrt{1-e^2\sin^2{\theta}} \ d\theta
    \end{align*}
    The above integral, known as elliptical integral of second kind, is non-elementary and hence, cannot be calculated by our commonly used methods. To calculate the integral, we use numerical approximation.
    \newpage
    \item If $\hat{a}$ and $\hat{b}$ are two unit vectors in $\bb{R}^2$ such that $\hat{a} \times \hat{b} \neq \vec{0}$. Show that $\{\hat{a}\cos{t} + \hat{b}\sin{t} \  | \ t \in [0, \ 2\pi] \}$ is an ellipse.
    \newline \hfill \newline
    Sol:
    Let $\hat{a} = a_1\hat{i} + a_2\hat{j}$ and $\hat{b} = b_1\hat{i}+b_2\hat{j}$
    \begin{align*}
        \hat{a}\cos{t} + \hat{b}\sin{t} &= (a_1\hat{i} + a_2\hat{j})\cos{t} + (b_1\hat{i}+b_2\hat{j})\sin{t} \\
        &= (a_1\cos{t} + b_1\sin{t})\hat{i} + (a_2\cos{t} + b_2\sin{t})\hat{j} \\
    \end{align*}
    Thus,
    \begin{align*}
        x &= a_1\cos{t} + b_1\sin{t} \\
        y &= a_2\cos{t} + b_2\sin{t} \\
    \end{align*}
    Solving the above equations for $\sin{t}$ and $\cos{t}$ we'll get:
    \begin{align*}
        \sin{t} &= \frac{a_1y-a_2x}{a_1b_2-a_2b_1}\\
        \cos{t} &= \frac{b_2x-b_1y}{a_1b_2-a_2b_1}\\
    \end{align*}
    Then,
    \begin{align*}
        \sin^2{t} + \cos^2{t} &= 1 \\
        \left ( \frac{a_1y-a_2x}{a_1b_2-a_2b_1} \right )^2 + \left( \frac{b_2x-b_1y}{a_1b_2-a_2b_1} \right)^2 &= 1
    \end{align*}
    \newpage
    \begin{align*}
        \frac{1}{(a_1b_2-a_2b_1)^2}[(a_1y-a_2x)^2 + (b_2x-b_1y)^2] &= 1 \\
        \frac{1}{(a_1b_2-a_2b_1)^2}[(a_1^2+b_1^2)y^2+(a_2^2+b_2^2)x^2-2(a_1a_2+b_1b_2)xy] &= 1
    \end{align*}
    The discriminant $\Delta$ of the above equation will be
    \begin{align*}
        \Delta &= (2(a_1a_2+b_1b_2))^2 - 4(a_1^2+b_1^2)(a_2^2+b_2^2) \\
        &= 4(2a_1a_2b_1b_2 - a_1^2b_2^2-a_2^2b_1^2) \\
        &= -4(a_1b_2-a_2b_1)^2 \leq 0
    \end{align*}
    Thus, the above equation describes an ellipse.
    \newpage
    \item 
\end{enumerate}

\end{document}
